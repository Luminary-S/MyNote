%%%%%%%%%%%%%%%%%%%%%%%%%%%%%%%%%%%%%%%%%
% kaobook
% LaTeX Template
% Version 1.0 (2/2/19)
%
% This template originates from:
% https://www.LaTeXTemplates.com
%
% Authors:
% Federico Marotta (federicomarotta@mail.com)
% Based on the doctoral thesis of Ken Arroyo Ohori (https://3d.bk.tudelft.nl/ken/en)
% and on the Tufte-LaTeX class.
% Modified for LaTeX Templates by Vel (vel@latextemplates.com)
%
% License:
% GPL Version 3 (see included LICENSE file)
%
%%%%%%%%%%%%%%%%%%%%%%%%%%%%%%%%%%%%%%%%%

%----------------------------------------------------------------------------------------
%	PACKAGES AND OTHER DOCUMENT CONFIGURATIONS
%----------------------------------------------------------------------------------------

\documentclass[
	fontsize=10pt, % Base font size
	twoside=false, % Use different layouts for even and odd pages (in particular, if twoside=true, the margin column will be always on the outside)
	%open=any, % If twoside=true, uncomment this to force new chapters to start on any page, not only on right pages
	%chapterprefix=true, % Uncomment to use the word "Chapter" before chapter numbers everywhere they appear
	%chapterentrydots=true, % Uncomment to output dots from the chapter name to the page number in the table of contents
	numbers=noenddot, % Comment to output dots after chapter numbers; the most common values for this option are: enddot, noenddot and auto (see the KOMAScript documentation for an in-depth explanation)
	%draft=true, % If uncommented, images will be replaced by empty boxes
	%overfullrule=true, % If uncommented, overly long lines will be marked by a black box
]{kaobook}

% Load common packages and commands
\usepackage{styles/environments}
\usepackage{styles/mdftheorems}
%\usepackage{styles/plaintheorems}

% Load packages for testing
\usepackage{blindtext}
\usepackage{subfigure}
%\usepackage{showframe}
%\usepackage{showlabels}

\graphicspath{{images/}{./}} % Paths in which to look for images

\addbibresource{main.bib} % Bibliography file

\makeindex[columns=3, title=Alphabetical Index, intoc] % Create an index

\makeglossaries % Create a glossary

\makenomenclature % Create nomenclature

%----------------------------------------------------------------------------------------

\begin{document}

%----------------------------------------------------------------------------------------
%	BOOK INFORMATION
%----------------------------------------------------------------------------------------

\titlehead{ Secret level: \textbf{Confidential} }
% \subject{Use this document as a template}

\title[Window Cleaning Robot {\normalfont\texttt{(WCR)}} Hardware Handbook ]{
	Window Cleaning Robot {\normalfont\texttt{(WCR)}} Hardware Handbook}
\subtitle{Version: 1.0}

\author[Guangli SUN]{ Guangli SUN }

\date{\today}

\publishers{CUHK CURI}

%----------------------------------------------------------------------------------------

\frontmatter % Denotes the start of the pre-document content, uses roman numerals

%----------------------------------------------------------------------------------------
%	OPENING PAGE
%----------------------------------------------------------------------------------------

%\makeatletter
%\extratitle{
%	% In the title page, the title is vspaced by 9.5\baselineskip
%	\vspace*{9\baselineskip}
%	\vspace*{\parskip}
%	\begin{center}
%		% In the title page, \huge is set after the komafont for title
%		\usekomafont{title}\huge\@title
%	\end{center}
%}
%\makeatother

%----------------------------------------------------------------------------------------
%	COPYRIGHT PAGE
%----------------------------------------------------------------------------------------

\makeatletter
\uppertitleback{\@titlehead} % Header

\lowertitleback{
	\textbf{Disclaimer}\\
	Forever in private reference.


	\medskip
	
	\textbf{Version History}\\
	\begin{enumerate}
		\item[*\textbf{1.0}] create the handbook, and add chapters of 
		force sensor, IMU, DC motors and encoder, Ultrasonic, A/D, stereo-cameras, mono-camera, battery
	\end{enumerate}
	
	% \cczero\ This book is released into the public domain using the CC0 code. To the extent possible under law, I waive all copyright and related or neighbouring rights to this work.
	
	% To view a copy of the CC0 code, visit: \\\url{http://creativecommons.org/publicdomain/zero/1.0/}
	
	\medskip
	
	\textbf{Colophon} \\
	This document was typeset with the help of \href{https://sourceforge.net/projects/koma-script/}{\KOMAScript} and \href{ttps://www.latex-project.org/}{\LaTeX} using the \href{https://github.com/fmarotta/kaobook/}{kaobook} class.
	
	The source code of this book is available at:\\\url{https://github.com/fmarotta/kaobook}
	
	(You are welcome to contribute!)
	
	% \medskip
	
	% \textbf{Publisher} \\
	% First printed in Jan 2019 by \@publishers
}
\makeatother

%----------------------------------------------------------------------------------------
%	DEDICATION
%----------------------------------------------------------------------------------------

\dedication{
	The harmony of the world is made manifest in Form and Number, and the heart and soul and all the poetry of Natural Philosophy are embodied in the concept of mathematical beauty.\\
	\flushright -- D'Arcy Wentworth Thompson
}

%----------------------------------------------------------------------------------------
%	TITLE PAGE
%----------------------------------------------------------------------------------------

% Note that \maketitle will actually print many pages.

% If twoside=false, \uppertitleback and \lowertitleback are not printed. To overcome this issue, we set twoside=semi just before printing the title pages, and set it back to false just after the title pages.
\KOMAoptions{twoside=semi}
\maketitle[3] % The [3] assigns "page 3" to the title, so that the cover page would get "page 1" (see KOMAScript documentation about maketitle)
\KOMAoptions{twoside=false}
% \maketitle
%----------------------------------------------------------------------------------------
%	PREFACE
%----------------------------------------------------------------------------------------

\chapter*{Version History}
\addcontentsline{toc}{chapter}{Version History}

\begin{enumerate}
    \item[*\textbf{1.0}] create the handbook, and add chapters of 
    force sensor, IMU, DC motors and encoder, Ultrasonic, A/D, stereo-cameras, mono-camera, battery
\end{enumerate}


\begin{flushright}
	\date{\today}
\end{flushright}

%----------------------------------------------------------------------------------------
%	TABLE OF CONTENTS & LIST OF FIGURES/TABLES
%----------------------------------------------------------------------------------------

\begingroup

% Define the style for the TOC, LOF, and LOT
%\setstretch{1}
%\hypersetup{linkcolor=DarkBlue}
\setlength{\textheight}{23cm}

% Turn on compatibility mode for the etoc package
\etocstandarddisplaystyle % "toc display" as if etoc was not loaded
\etocstandardlines % "toc lines as if etoc was not loaded

\tableofcontents % Output the table of contents

\listoffigures % Output the list of figures

% Comment both of the following lines to have the LOF and the LOT on different pages
\let\cleardoublepage\bigskip
\let\clearpage\bigskip

\listoftables % Output the list of tables

\endgroup

%----------------------------------------------------------------------------------------
%	MAIN BODY
%----------------------------------------------------------------------------------------

\mainmatter % Denotes the start of the main document content, resets page numbering and uses arabic numbers

\setchapterstyle{kao}
\setchapterpreamble[u]{\margintoc}
\chapter{Introduction}
\labch{intro}

\section{The main ideas}
The book is in the following arrangement:
\begin{figure*}[h!]
	\includegraphics{01_hardware_all.png}
	\caption[Hardware structure]{Hardware structure of Window Cleaning Robot}
\end{figure*}


\begin{description}
	\item[style.sty] This package contains the specifications of page 
	layout, headers and footers, chapter headings, and the fonts used 
	throughout the document.
	\item[packages.sty] Loads additional packages to decorate the 
	writing with special contents (for instance, the \Package{listing} 
	package is loaded here as it is not required in every book). There 
	are also defined some useful commands to print the same words always 
	in the same way, \eg latin words in italics or \Package{packages} in 
	verbatim.
	\item[references.sty] Some useful commands to manage labeling and 
	referencing, again to ensure that the same elements are referenced 
	always in a consistent way.
	\item[environments.sty] Provides special environments, like boxes. 
	Both simple and complex environments are available; by complex we 
	mean that they are endowed with a counter, floating and can be put 
	in a special table of contents.\sidenote[-2mm][]{See 
	\vrefch{mathematics} for some examples.}
	\item[theorems.sty] The style of mathematical environments. 
	Acutally, there are two such packages: one is for plain theorems, 
	\ie the theorems are printed in plain text; the other uses 
	\Package{mdframed} to draw a box around theorems. You can plug the 
	most appropriate style into its document.
\end{description}



%-------------------------------
%	S1: base sensors
%--------------------------------
\pagelayout{wide} % No margins
\addpart{ Base Sensors }
\pagelayout{margin} % Restore margins
\setchapterstyle{kao}
\setchapterpreamble[u]{\margintoc}
\chapter{IMU}
\labch{11IMU}

\section{Function}
The IMU is sale by RuiFen Technology.

\setchapterstyle{kao}
\setchapterpreamble[u]{\margintoc}
\chapter{Stereo Camera -- ZED1}
\labch{12zed}

\section{Function}
The stereo camera here is ZED1
\input{chapters/13_Ultrasonic.tex}
\input{chapters/14_SonicMotor.tex}
\input{chapters/15_Servo.tex}
\input{chapters/16_Laser.tex}

% \input{chapters/options.tex}
% \input{chapters/textnotes.tex}
% \input{chapters/figsntabs.tex}
% \input{chapters/references.tex}

%-------------------------------
%	S2: head sensors
%--------------------------------
\pagelayout{wide} % No margins
\addpart{ Head Sensors }
\pagelayout{margin} % Restore margins
\input{chapters/21_ForceSensor.tex}
\input{chapters/22_HeadCam.tex}
% \input{chapters/layout.tex}
% \input{chapters/mathematics.tex}

%-------------------------------
%	S3: Main Actuators
%--------------------------------
\pagelayout{wide} % No margins
\addpart{ Main Acuators }
\pagelayout{margin} % Restore margins
\input{chapters/31_UR.tex}
\input{chapters/32_RCRMotor.tex}
\setchapterstyle{kao}
\setchapterpreamble[u]{\margintoc}
\chapter{Rope-Climbing Robot Incremental Encoder}
\labch{33en}

\section{Function}
It is used for measuring the position and speed of the rope-climbing robot.
\setchapterstyle{kao}
\setchapterpreamble[u]{\margintoc}
\chapter{RoboModule Motor Driver}
\labch{34RMD}

\section{Function Description}
The RoboModule Motor Driver is used for the controlling of Rope-Climbing Robot climbing motor, it should be 
used with Can Analysis \ref{ch:51CA}. 
% \sidenote{  \href{https://www.akm.com/cn/zh-cn/technology/technical-tutorial/basic-knowledge-magnetic-sensor/hall-sensors/}
% {Introduction of Hall sensors link} }. 
% \begin{figure}[hb]
% 	\includegraphics[width=0.45\textwidth]{4101_motor1.jpg}
% 	\caption[Mona Lisa, again]{It's Mona Lisa again.}
% 	\labfig{fig411_cleaningMotor}
% \end{figure}

Application: large current BLDC, replace epos, elmo and minglang; RoboMasters.

\begin{figure}[htb]	
	\centering
	\begin{subfigure}
		\centering
		\includegraphics[width=2in]{3401_motordriver.jpg}
		% \caption{motor with reduction detail}\label{fig:4101}		
	\end{subfigure}
	\quad
	\begin{subfigure}
		\centering
		\includegraphics{3402_motordriver.jpg}
		% \caption{motor with brand}\label{fig:4102}
	\end{subfigure}
	% \begin{subfigure}
	% 	\centering
	% 	\includegraphics[width=2in]{412_hallencoder.jpg}
	% 	% \caption{motor with brand}\label{fig:4102}
	% \end{subfigure}
	\caption[RoboModule Motor Driver]{ 
		RoboModule Motor Driver		
			}\label{fig:340}
\end{figure}
% \marginnote[-12pt]{
	
% 	}
\section{Purchase Info}
Purchase:
\begin{enumerate}
	\item brand: RoboModule
	\item type: RMDS-403 and RMDS-109
	\item link: \href{https://item.taobao.com/item.htm?id=43995228864}{taobao link} 
	\item price: 700RMB and 370RMB
	\item data: June. 2020
	\item RP: SGL, clear
\end{enumerate}

Supporting Material:
\begin{enumerate}
	\item RoboModule website: \url{http://www.robomodule.net/download.html}
    \item Always refer to: online instructions of 003(common problems) and 011 (can protocol), 
      They have been downloaded in ros package robomodule sheet folder on {\color{red} September 19, 2020.}.
	\item telephone: +86-18503054370
    \item rs232 serial wire purchase link: \url{https://detail.tmall.com/item.htm?id=39113690170 }
\end{enumerate}


\section{Instruction for Connection}
\begin{marginfigure}[-2cm]
	\includegraphics[width=0.8\textwidth]{3434_RMDS_cn_can2controller.png}
	\caption[arduino ttl to can ]{arduino ttl to can }
	\labfig{fig3434:can2controller}
\end{marginfigure}
\subsection{Windows}

In windows,  followings should be done:
\begin{enumerate}
	\item set feedback part (encoder) parameters, write in encoder resolution.
	\item adjust motor and encoder direction, using GUI velocity mode, check pop data and pwm set data direction, if not same, change direction of the Motor.
    \item check the using mode in the GUI, velocity and velocity-position mode are using in our program.
    \item set the driver's group and id number. here {\color{red}403: G0, id1; 109: G1, id1}.
\end{enumerate}
\begin{marginfigure}
	\includegraphics[width=1\textwidth]{3432_RMDS_cn_232pins.png}
	\caption[RS232 Pins]{RS232 Pins connection: {\color{red} 2:232T; 3:232R; 5:GND.} }
	\labfig{fig3432:232pins}
\end{marginfigure}

\begin{marginfigure}
	\includegraphics[width=1\textwidth]{3433_RMDS_cn_arduino_ttl_can.png}
	\caption[arduino ttl to can ]{arduino ttl to can }
	\labfig{fig3433:ttltocan}
\end{marginfigure}

communication rate:
\begin{enumerate}
	\item can: 1000khz
	\item 232: 115200
	\item 485: 115200
\end{enumerate}

\begin{figure}[htb]
	\includegraphics[width=1\textwidth]{3431_RMDS_cn_windows_tune.png}
	\caption[tune wire connection]{ 
		tune wire connection.		 
		}
	\labfig{fig3431:windows_tune}
\end{figure}
% 1234

% wire definition in encoders:
% \begin{enumerate}
% 	\item {\colorbox{red}W red}: M1, motor + positive
% 	\item {\colorbox{black}W black}: GND, encoder power supply - negative
% 	\item {\colorbox{yellow}W yellow}: C1, encoder signal A phase
% 	\item {\colorbox{green}W green}: C2, encoder signal B phase
% 	\item {\colorbox{blue}W blue}: 3.3/5v, encoder power supply + positive
% 	\item {\colorbox{white}W white}: M2, motor - negative 
% \end{enumerate}

\section{Mechanical Specification}
It includes magnetic motor main part, reduction,  D shaft and a magnetic incremental encoder.

Motor specification:
\begin{enumerate}
	\item RMDS109:  W: 56mm, L: 96mm; H: 22.5mm
	\item RMDS403: W: 76mm, L: 118mm; H: 33mm
\end{enumerate}

\begin{figure}[htb]
	\includegraphics[width=0.8\textwidth]{3441_RMDS_109_mech_install.png}
	\caption[RMDS109 mechanical installation]{ 
		RMDS109 mechanical installation. (109 is same as 108), in 403,401, just care 87mm to 105mm		 
		}
	\labfig{fig3441_mech_install}
\end{figure}

\section{Electrical Specification}
RMDS specification:
\begin{marginfigure}[-3cm]
	\includegraphics[width=1\textwidth]{343_RMDS_el_sp.png}
	\caption[RMDS Driver electrical specification]{ 
		RMDS Driver electrical specification, here we have used the formal version of them, 403 and 109.		 
		}
	\labfig{fig:343}
\end{marginfigure}
RMDS109:
\begin{enumerate}
	\item power supply: 7 \textasciitilde 33v
	\item constant output current: 10A
	\item max speed: -32768 \textasciitilde +32767 RPM
\end{enumerate}

RMDS403:
\begin{enumerate}
	\item power supply: 10 \textasciitilde 58v
	\item constant output current: 30A
	\item max speed: -32768 \textasciitilde +32767 RPM
\end{enumerate}


\section{others}
\subsection{how to use 12v motor in 24v power supply }
set in the windows GUI, set PWM and current limitation.
\begin{figure}[!htb]
	\includegraphics[width=0.8\textwidth]{345_use24vfor12vmotor.png}
	\caption[how to use 12v motor in 24v power supply]{ 
		how to use 12v motor in 24v power supply.		 
		}
	\labfig{fig:345}
\end{figure}

\subsection{how to judge differential encoder and single port encoder }
two AB or single AB
\begin{figure}[!htb]
	\includegraphics[width=0.5\textwidth]{344_diff_encoder_judge.png}
	\caption[how to judge differential encoder and single port encoder]{ 
		how to judge differential encoder and single port encoder.		 
		}
	\labfig{fig:344}
\end{figure}

\section{Protocol}
mode value:
\begin{table}[htb!]
	\caption[RoboModule mode value]{RoboModule mode value table.}
	\labtab{341_rmv}
	\begin{tabular}{ c c }
		\toprule
		mode name & DATA[0] in init cmd \\
		\midrule
		open loop  & 0x01  \\
		\midrule
		current  & 0x02  \\
		\midrule
		velocity  & 0x03  \\
		\midrule
		position  & 0x04  \\
		\midrule
		vel-position  & 0x05 \\
		\midrule
		current-vel  & 0x06  \\
		\midrule
		current-position  & 0x07  \\
		\midrule
		current-vel-pos  & 0x08  \\
		\bottomrule
	\end{tabular}
\end{table}

command index for each mode is the value of mode value + 1;

%-------------------------------
%	S4: Assisted Actuators
%--------------------------------
\pagelayout{wide} % No margins
\addpart{ Assisted Acuators }
\pagelayout{margin} % Restore margins
\setchapterstyle{kao}
\setchapterpreamble[u]{\margintoc}
\chapter{Cleaning Motor}
\labch{41CM}

\section{Function Description}
The cleaning motor is installed in the head cleaning unit, it drives the brush to rotating in designed speed.
({\color{red} speed control}).
It includes a BLDC (brushless direct current) motor and a magnetic Hall incremental encoder 
\sidenote{  \href{https://www.akm.com/cn/zh-cn/technology/technical-tutorial/basic-knowledge-magnetic-sensor/hall-sensors/}
{Introduction of Hall sensors link} }. 
% \begin{figure}[hb]
% 	\includegraphics[width=0.45\textwidth]{4101_motor1.jpg}
% 	\caption[Mona Lisa, again]{It's Mona Lisa again.}
% 	\labfig{fig411_cleaningMotor}
% \end{figure}

Application: wheels(toy tire 60mm also supplied), camera swinging arm motor.

\begin{figure}[htb]	
	\centering
	\begin{subfigure}
		\centering
		\includegraphics[width=2in]{4101_motor.jpg}
		% \caption{motor with reduction detail}\label{fig:4101}		
	\end{subfigure}
	\quad
	\begin{subfigure}
		\centering
		\includegraphics[width=2in]{4102_motor.jpg}
		% \caption{motor with brand}\label{fig:4102}
	\end{subfigure}
	\begin{subfigure}
		\centering
		\includegraphics[width=2in]{412_hallencoder.jpg}
		% \caption{motor with brand}\label{fig:4102}
	\end{subfigure}
	\caption[Cleaning Motor]{ 
		Cleaning Motor: BLDC motor with Magnetic incremental hall encoder in cleaning unit.
		\href{https://www.ebay.com/itm/370-Motor-Hall-Encoder-DC-2-5V-24V-12-PPR-Dual-Quadrature-Outputs-Metal-Encoder/283429506020}
{370 magnetic Hall incremental encoder(MHIE) Pic Source} (here we use 520 MHIE) 		
			}\label{fig:410}
\end{figure}
% \marginnote[-12pt]{
% 	}

\section{Purchase Info}
\begin{enumerate}
	\item brand: CHIHAI MOTOR
	\item type: JGB37-520
	\item link: \href{https://item.taobao.com/item.htm?spm=a1z09.2.0.0.4a892e8d2d0m2a&id=531752422073&_u=i1s32jc02624}{taobao link} 
	\item price:  60RMB
	\item data: June. 2020
	\item RP: CZ, clear
\end{enumerate}

\section{Instruction for Connection}
wire definition in encoders:
\begin{enumerate}
	\item {\colorbox{red}W red}: M1, motor + positive
	\item {\colorbox{black}W black}: GND, encoder power supply - negative
	\item {\colorbox{yellow}W yellow}: C1, encoder signal A phase
	\item {\colorbox{green}W green}: C2, encoder signal B phase
	\item {\colorbox{blue}W blue}: 3.3/5v, encoder power supply + positive
	\item {\colorbox{white}W white}: M2, motor - negative 
\end{enumerate}

\section{Mechanical Specification}
It includes magnetic motor main part, reduction,  D shaft and a magnetic incremental encoder.

Motor specification:
\begin{enumerate}
	\item D shaft diameter: 6mm
	\item shaft length: 21mm, D part length: 12mm
	\item weight: 160g (different with the reduction ratio)
	\item L=24mm here
\end{enumerate}

\begin{figure}[htb]
	\includegraphics[width=1\textwidth]{411_motor_mechanical.jpg}
	\caption[Cleaning Motor mechanical specification]{ 
		Cleaning Motor electrical specification.		 
		}
	\labfig{fig412_hallencoder}
\end{figure}

\section{Electrical Specification}
Motor specification:
\begin{enumerate}
	\item voltage: 12v
	\item reduction ratio: 90 
	\item encoder resolution: 11* reduction ratio = 11*90=990
	\item speed w/o payload: 107rpm
	\item maximum output power: 7w
	\item nominal torque: 1.4N*m
\end{enumerate}

Encoder specification:
\begin{enumerate}
	\item type: AB biple phase incremental magnetic hall encoder
	\item power supply: 3.3v/5V
	\item frequency: 100KHZ
	\item basic pulse number: 11 PPR
	\item output signal: AB phase square wave
	\item interface: PH2.0
	\item directly connected to atom controller, self-included pulling resist
\end{enumerate}
\begin{figure}[htb]
	\includegraphics[width=0.8\textwidth]{413_el_specification}
	\caption[Cleaning Motor electrical specification]{ 
		Cleaning Motor electrical specification, here voltage is 12v, reduction ratio is 90.		 
		}
	\labfig{fig:413}
\end{figure}

\input{chapters/42_Pump.tex}
\setchapterstyle{kao}
\setchapterpreamble[u]{\margintoc}
\chapter{ Relay 485 Module}
\labch{43R4M}

\section{Function Description}
4 roads IO relay control board with 485 protocal to USB is used for sewage and clean pump open and close control. 


Application: 24v io control.

\begin{figure}[!htb]
	\includegraphics[width=1\textwidth]{43_R.jpg}
	\caption[485 Relay module]{ 
		485 Relay module.		 
		}
	\labfig{fig43_R}
\end{figure}
% \marginnote[-12pt]{
	
% 	}

\section{Purchase Info}
\begin{enumerate}
	\item brand: KMCZE
	\item type: KMCZE-I4O4-U241.0
	\item link: \href{https://item.taobao.com/item.htm?id=559200622128}{taobao link} 
	\item price: 155RMB
	\item data: June. 2020
	\item RP: SGL, clear
\end{enumerate}


\section{Electrical Specification}

\begin{figure}[!htb]
	\includegraphics[width=1\textwidth]{433_R_sp.png}
	\caption[ relay electrical specification]{ 
		relay electrical specification, here 4 roads is bought.		 
		}
	\labfig{fig:433_R_sp}
\end{figure}

\begin{enumerate}
	\item power supply: 24v
	\item constant output current: 10A
\end{enumerate}


%-------------------------------
%	S5: Communication
%--------------------------------
\pagelayout{wide} % No margins
\addpart{ Communication }
\pagelayout{margin} % Restore margins
\setchapterstyle{kao}
\setchapterpreamble[u]{\margintoc}
\chapter{Can Analysis} 
\labch{51CA}

\section{Function Description}
The ZLG CANOpen Analysis is used for the controlling of Rope-Climbing Robot climbing motor, it should be 
used with Can Analysis \ref{ch:34RMD}. 
% \sidenote{  \href{https://www.akm.com/cn/zh-cn/technology/technical-tutorial/basic-knowledge-magnetic-sensor/hall-sensors/}
% {Introduction of Hall sensors link} }. 
% \begin{figure}[hb]
% 	\includegraphics[width=0.45\textwidth]{4101_motor1.jpg}
% 	\caption[Mona Lisa, again]{It's Mona Lisa again.}
% 	\labfig{fig411_cleaningMotor}
% \end{figure}

Application: CANOpen protocol analysis.

\begin{figure}[htb]	
	\centering
	\begin{subfigure}
		\centering
		\includegraphics[width=2in]{5111_CA_I_mini.png}
		% \caption{motor with reduction detail}\label{fig:4101}		
	\end{subfigure}
	\quad
	\begin{subfigure}
		\centering
		\includegraphics[width=2in]{5111_CA_II.jpg}
		% \caption{motor with brand}\label{fig:4102}
	\end{subfigure}
	% \begin{subfigure}
	% 	\centering
	% 	\includegraphics[width=2in]{412_hallencoder.jpg}
	% 	% \caption{motor with brand}\label{fig:4102}
	% \end{subfigure}
	\caption[ZLG CANOpen Analysis]{ 
		ZLG CANOpen Analysis	
			}\label{fig:510}
\end{figure}
% \marginnote[-12pt]{
	
% 	}
\section{Purchase Info}
Purchase:
\begin{enumerate}
	\item brand: ZLG(zhiyuan electrical)
	\item type: USBCAN-I-MINI
	\item link: \href{https://item.taobao.com/item.htm?id=582701516953&price=1199}{taobao link} 
	\item price: 1199RMB
	\item data: June. 2020
	\item RP: SGL, clear
\end{enumerate}

Supporting Material:
\begin{enumerate}
	\item ZLG download website: \url{https://www.zlg.cn/can/down/down/id/22.html}
    \item Always refer to: online instructions of USBCAN-I-MINI instruction, \url{https://manual.zlg.cn/web/#/63?page_id=2568} 
      They have been downloaded in ros package robomodule sheet folder on {\color{red} September 19, 2020.}.
	% \item telephone: +86-18503054370
    % \item rs232 serial wire purchase link: \url{https://detail.tmall.com/item.htm?id=39113690170 }
\end{enumerate}


\section{Instruction for Connection}

\begin{figure}[htb]
	\includegraphics[width=1\textwidth]{5131_CA_DP9_canpin.png}
	\caption[Can Analysis DP9 pin]{ 
		Can Analysis DP9 pin.		 
		}
	\labfig{fig5131:CA}
\end{figure}
% 1234

% wire definition in encoders:
% \begin{enumerate}
% 	\item {\colorbox{red}W red}: M1, motor + positive
% 	\item {\colorbox{black}W black}: GND, encoder power supply - negative
% 	\item {\colorbox{yellow}W yellow}: C1, encoder signal A phase
% 	\item {\colorbox{green}W green}: C2, encoder signal B phase
% 	\item {\colorbox{blue}W blue}: 3.3/5v, encoder power supply + positive
% 	\item {\colorbox{white}W white}: M2, motor - negative 
% \end{enumerate}

\section{Mechanical Specification}

\begin{enumerate}
	\item USBCAN-I-mini:  W: 59mm, L: 75.5mm; H: 14mm
\end{enumerate}

% \begin{figure}[htb]
% 	\includegraphics[width=0.8\textwidth]{3441_RMDS_109_mech_install.png}
% 	\caption[RMDS109 mechanical installation]{ 
% 		RMDS109 mechanical installation. (109 is same as 108), in 403,401, just care 87mm to 105mm		 
% 		}
% 	\labfig{fig3441_mech_install}
% \end{figure}

\section{Interface Library Notes}
\begin{description}
	\item[Device type] Different version of ZLG CAN has different device type, it should be set in the \textbf{demo\_config.yaml}.
	\item[Baudrate setup] Different CAN Baudrate should be set in the function: \textit{VCI\_INIT\_CONFIG}, in our file \textbf{can\_application.cpp}
% function: $\textit{VCI_INIT_CONFIG}$


\end{description}

\begin{table}[htb!]
	\caption[Device type number table]{Device type number table.}
	\labtab{511Candevicenumber}
	\begin{tabular}{ c c c }
		\toprule
		product type & library device name & device type number \\
		\midrule
		USBCAN-I/I+  & USBCAN1  &   3 \\
		\midrule
		USBCAN-II/II+ & USBCAN2 & 4 \\
		% \multirow{3}{4em}{Multiple row} & cell2 & cell3 & cell4\\ &
		% cell5 & cell6 & cell7 \\ &
		% cell8 & cell9 & cell10 \\
		% \multirow{3}{4em}{Multiple row} & cell2 & cell3 & cell4 \\ &
		% cell5 & cell6 & cell7 \\ &
		% cell8 & cell9 & cell10 \\
		\bottomrule
	\end{tabular}
\end{table}

\begin{table}[htb!]
	\caption[Can Analysis Baudrate timing set table]{Can Analysis Baudrate timing set table.}
	\labtab{512Baudrate}
	\begin{tabular}{ c c c }
		\toprule
		Can Baudrate & Timeing0 & Timing1 \\
		\midrule
		500Kbps  & 0x00  &   0x1C \\
		\midrule
		1000Kbps  & 0x00  &   0x14 \\
		\bottomrule
	\end{tabular}
\end{table}
% \section{others}
% \subsection{how to use 12v motor in 24v power supply }
% set in the windows GUI, set PWM and current limitation.
% \begin{figure}[!htb]
% 	\includegraphics[width=0.8\textwidth]{345_use24vfor12vmotor.png}
% 	\caption[how to use 12v motor in 24v power supply]{ 
% 		how to use 12v motor in 24v power supply.		 
% 		}
% 	\labfig{fig:345}
% \end{figure}




\setchapterstyle{kao}
\setchapterpreamble[u]{\margintoc}
\chapter{A\/D}
\labch{52AD}

\section{Function Description}
The cleaning motor is installed in the head cleaning unit, it drives the brush to rotating in designed speed.
({\color{red} speed control}).
It includes a BLDC (brushless direct current) motor and a magnetic Hall incremental encoder 


Application: wheels(toy tire 60mm also supplied), camera swinging arm motor.

% \begin{figure}[!htb]
% 	\includegraphics[width=1\textwidth]{520_AD.jpg}
% 	\caption[AD module]{ 
% 		AD module.		 
% 		}
% 	\labfig{fig520_AD.jpg}
% \end{figure}
% \marginnote[-12pt]{
	
% 	}

\section{Purchase Info}
\begin{enumerate}
	\item brand: KMCZE
	\item type: KMCZE-I4O4-U241.0
	\item link: \href{https://item.taobao.com/item.htm?id=559200622128}{taobao link} 
	\item price: 700RMB and 370RMB
	\item data: June. 2020
	\item RP: CZ, clear
\end{enumerate}



%-------------------------------
%	S6: PC and MCU
%--------------------------------
\pagelayout{wide} % No margins
\addpart{ PC and MCU }
\pagelayout{margin} % Restore margins
\input{chapters/61_PC.tex}
\input{chapters/62_MCU.tex}
% \appendix % From here onwards, chapters are numbered with letters, as is the appendix convention

% \pagelayout{wide} % No margins
% \addpart{Appendix}
% \pagelayout{margin} % Restore margins

% \input{chapters/appendix.tex}

% %----------------------------------------------------------------------------------------

% \backmatter % Denotes the end of the main document content

% \setchapterstyle{plain} % Output plain chapters from this point onwards

% %----------------------------------------------------------------------------------------
% %	BIBLIOGRAPHY
% %----------------------------------------------------------------------------------------

% % The bibliography needs to be compiled on the command line with 'biber main' from the template directory

% \defbibnote{bibnote}{Here are the references in citation order.\par\bigskip} % Prepend this text to the bibliography
% \printbibliography[heading=bibintoc, title=Bibliography, prenote=bibnote] % Add the bibliography heading to the ToC and set the title of the bibliography

% %----------------------------------------------------------------------------------------
% %	NOMENCLATURE
% %----------------------------------------------------------------------------------------

% % The nomenclature needs to be compiled on the command line with 'makeindex main.nlo -s nomencl.ist -o main.nls' from the template directory

% \nomenclature{$c$}{Speed of light in a vacuum inertial frame}
% \nomenclature{$h$}{Planck constant}

% \renewcommand{\nomname}{Notation}
% \renewcommand{\nompreamble}{The next list describes several symbols that will be later used within the body of the document.}
% \printnomenclature % Output the nomenclature

% %----------------------------------------------------------------------------------------
% %	GREEK ALPHABET
% % 	Originally from https://gitlab.com/jim.hefferon/linear-algebra
% %----------------------------------------------------------------------------------------

% \vspace{3cm}
% {\usekomafont{chapter}Greek letters with pronounciation} \\[2ex]
% \begin{center}
% 	\newcommand{\pronounced}[1]{\hspace*{.2em}\small\textit{#1}}
% 	\begin{tabular}{l l @{\hspace*{3em}} l l}
% 		\toprule
% 		Character & Name & Character & Name \\ 
% 		\midrule
% 		$\alpha$ & alpha \pronounced{AL-fuh} & $\nu$ & nu \pronounced{NEW} \\
% 		$\beta$ & beta \pronounced{BAY-tuh} & $\xi$, $\Xi$ & xi \pronounced{KSIGH} \\ 
% 		$\gamma$, $\Gamma$ & gamma \pronounced{GAM-muh} & o & omicron \pronounced{OM-uh-CRON} \\
% 		$\delta$, $\Delta$ & delta \pronounced{DEL-tuh} & $\pi$, $\Pi$ & pi \pronounced{PIE} \\
% 		$\epsilon$ & epsilon \pronounced{EP-suh-lon} & $\rho$ & rho \pronounced{ROW} \\
% 		$\zeta$ & zeta \pronounced{ZAY-tuh} & $\sigma$, $\Sigma$ & sigma \pronounced{SIG-muh} \\
% 		$\eta$ & eta \pronounced{AY-tuh} & $\tau$ & tau \pronounced{TOW (as in cow)} \\
% 		$\theta$, $\Theta$ & theta \pronounced{THAY-tuh} & $\upsilon$, $\Upsilon$ & upsilon \pronounced{OOP-suh-LON} \\
% 		$\iota$ & iota \pronounced{eye-OH-tuh} & $\phi$, $\Phi$ & phi \pronounced{FEE, or FI (as in hi)} \\
% 		$\kappa$ & kappa \pronounced{KAP-uh} & $\chi$ & chi \pronounced{KI (as in hi)} \\
% 		$\lambda$, $\Lambda$ & lambda \pronounced{LAM-duh} & $\psi$, $\Psi$ & psi \pronounced{SIGH, or PSIGH} \\
% 		$\mu$ & mu \pronounced{MEW} & $\omega$, $\Omega$ & omega \pronounced{oh-MAY-guh} \\
% 		\bottomrule
% 	\end{tabular} \\[1.5ex]
% 	Capitals shown are the ones that differ from Roman capitals.
% \end{center}

% %----------------------------------------------------------------------------------------
% %	GLOSSARY
% %----------------------------------------------------------------------------------------

% % The glossary needs to be compiled on the command line with 'makeglossaries main' from the template directory

% \newglossaryentry{computer}{
% 	name=computer,
% 	description={is a programmable machine that receives input, stores and manipulates data, and provides output in a useful format}
% }

% \newacronym[longplural={Frames per Second}]{fpsLabel}{FPS}{Frame per Second}
% \newacronym[longplural={Tables of Contents}]{tocLabel}{TOC}{Table of Contents}

% \setglossarystyle{listgroup} % Set the style of the glossary (see https://en.wikibooks.org/wiki/LaTeX/Glossary for a reference)
% \printglossary[title=Special Terms, toctitle=List of terms] % Output the glossary, 'title'  is the chapter heading for the glossary, toctitle is the table of contents heading

% %----------------------------------------------------------------------------------------
% %	INDEX
% %----------------------------------------------------------------------------------------

% % The index needs to be compiled on the command line with 'makeindex main' from the template directory

% \printindex % Output the index

%----------------------------------------------------------------------------------------
%	BACK COVER
%----------------------------------------------------------------------------------------

% If you have a PDF file that you want to use as back cover, uncomment the following lines.

%\clearpage
%\thispagestyle{empty}
%\null%
%\clearpage
%\includepdf{cover-back.pdf}

%----------------------------------------------------------------------------------------

\end{document}
