\chapter{A Brief History of \texorpdfstring{\LaTeX}{LaTeX}}

\label{history}

Donald Knuth is celebrated among programmers as
the man who coined the term \emph{analysis of algorithms}
and pioneered many computer science fundamentals we use today.
Knuth is perhaps most famous for his ongoing magnum opus,
\textit{The~Art of Computer Programming}.

When the first volume of \acronym{taocp} was released in 1968,
it was printed the same way most books had been since the turn of the century:
with \introduce{hot metal} type.
Letters were cast from molten lead,
then arranged into lines.
These lines were clamped together to form pages,
which were inked and pressed against paper.

By March of 1977, Knuth was ready for a second run of \acronym{taocp}, volume~2,
but he was horrified when he received the proofs.
Hot metal typesetting was expensive, complicated, and time-consuming,
so publishers had replaced it with phototypesetting,
which works by projecting images of characters onto film.
The new technology, while much cheaper and faster,
didn't provide the quality Knuth
expected.\punckern\endnote{Knuth, \textit{Digital Typography} (Stanford, 1999), 3--5}

The average author would have resigned themselves to this change and moved on,
but Knuth took great pride in his books' appearances,
especially for their mathematics.
Around this time, he also discovered the growing field of digital typesetting,
where glyphs are built from tiny dots,
packed together at over 1,000 per inch.
Inspired,
Knuth set off on one of the greatest yak shaves\footnote{Programmers
call seemingly unrelated work needed to solve their main problem
``yak shaving''\quotekern. The phrase is thought to originate from an episode
of \textit{The Ren~\&~Stimpy Show}.\punckern\endnote{``yak shaving''\quotekern,
\textit{The Jargon File},
\href{http://www.catb.org/~esr/jargon/html/Y/yak-shaving.html}%
{\texttt{www.catb.org/\~{}esr/jargon/html/Y/yak-shaving.html}}}}
of all time.
For years, he paused work on his books to create his own
typesetting system.
When the dust settled in 1978, he had the first version of
\TeX.\punckern\footnote{The name ``\TeX{}'' comes from the Greek
{\fontspec[Scale=MatchLowercase]{NotoSerif-Medium}τέχνη},
meaning \introduce{art} or \introduce{craft}.\punckern\endnote{Knuth,
\textit{The \TeX book}, 1}}

It's hard to appreciate how much of a revolution \TeX{} was,
especially looking back from a time where anybody with a copy
of Word can be their own desktop publisher.
Adobe's \acronym{pdf} wouldn't exist for another decade, so Knuth
and his graduate students devised their own device-independent format,
\acronym{dvi}.
Scalable fonts were uncommon, so he created \MF{} to rasterize glyphs
into dots on the page.
Perhaps most importantly, Knuth and his students designed algorithms
to automatically hyphenate and justify text into
beautifully-typeset paragraphs.\punckern\footnote{These same algorithms went
on to influence the ones Adobe uses in its software today.\punckern\endnote{%
Several sources (\http{www.tug.org/whatis.html},
\https{tug.org/interviews/thanh.html},
\http{www.typophile.com/node/34620})
mention \TeX's influence on the \textit{hz}-program by Peter Karow
and Hermann Zapf, thanks to via Knuth's collaborations with Zapf.
\textit{hz} was later acquired by Adobe and used
when creating InDesign's paragraph formatting systems.}}

\LaTeX{}, short for Lamport~\TeX{}, was later developed by Leslie Lamport
as a set of commands for common document layouts.
It was introduced in 1986 with his guide,
\textit{\LaTeX: A~Document Preparation System}.
Other typesetting systems based on \TeX{} also exist,
the other most popular today being Con\TeX{}t.

Development continues today,
both in the form of user-provided packages for \TeX{} and \LaTeX{},
and as improvements to the \TeX{} typesetting program itself.
There are four versions, or \introduce{engines}:
\begin{description}
\item[\TeX] is the original system by Donald Knuth.
Knuth stopped adding features after version 3.0 in March~1990,
and all subsequent releases have contained only bug fixes.
With each release, the version number asymptotically approaches $\pi$
by adding an additional digit.
The most recent version, 3.14159265, came out in January~2014.

\item[pdf\TeX] is an extension of \TeX{} that provides direct \acronym{pdf}
    output (instead of \TeX's \acronym{dvi}),
    native support for PostScript
    and TrueType fonts,
    and micro-typographic features discussed in \chapref{microtype}.
    It was originally developed by
    Hàn Thế Thành
    as part of his PhD thesis
    for Masaryk University in Brno, Czech Republic.\punckern\endnote{%
    Hàn Thế Thành,
    \textit{Micro-typographic extensions to the \TeX{} typesetting system}
    (Masaryk University Brno, October 2000)}

\item[\XeTeX] is a further extension of \TeX{} that adds native support for
    Unicode and OpenType.
    It was originally developed by Jonathan Kew in the early 2000s,
    and gained full cross-platform support in 2007.\punckern\endnote{Jonathan Kew,
    ``\XeTeX{} Live''\quotekern, \textit{TUGboat} 29, no.~1 (2007)}

\item[\LuaTeX] is similar to \XeTeX{} in its native Unicode and modern font support.
    It also embeds the Lua scripting language into the engine,
    exposing an interface for package and document authors.
    It first appeared in 2007 and is developed by a core team of
    Hans Hagen, Hartmut Henkel, Taco Hoekwater,
    and Luigi Scarso.\punckern\endnote{\http{www.luatex.org}}
\end{description}

Building \TeX{} today is an\dots{} interesting endeavor.
When it was written in the late 1970s,
there were no large, well-documented, open-source projects for students to study,
so Knuth set out to make \TeX{} into one.
As part of this effort, \TeX{} was written in a style he calls
\introduce{literate programming}: opposite most programs---where
documentation is interspersed throughout the code---Knuth wrote \TeX{} as a book,
with the code inserted between paragraphs.
This mix of English and code is called \texttt{WEB}.\punckern\footnote{Knuth
also released a pair of companion programs named
\texttt{TANGLE} and \texttt{WEAVE}.
The former extracts the book---as \TeX, of course---and the latter
produces \TeX's Pascal source code.}

Unsurprisingly, most modern systems don't have good tooling for the late 1970s
dialect of Pascal that \TeX{} was written in,
so present-day distributions use another program,
\texttt{web2c}, to convert its \texttt{WEB} source into C code.
pdf\TeX{} and \XeTeX{} are built by combining the result with other C
and \cpp{} sources.
Instead of taking this complicated approach,
the \LuaTeX{} authors hand-translated Knuth's Pascal into C.
They have used the resulting code since 2009.\punckern\endnote{%
Taco Hoekwater, \textit{\LuaTeX{} says goodbye to Pascal}
(MAPS 39, Euro\TeX{} 2009),
\https{www.tug.org/TUGboat/tb30-3/tb96hoekwater-pascal.pdf}}

