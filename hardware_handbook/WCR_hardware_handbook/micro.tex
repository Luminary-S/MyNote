\chapter{Microtypography}
\label{microtype}

\introduce{Microtypography} is the craft of improving a document's legibility
with small, subliminal tweaks.
In other words, it is
\begin{quote}
\small
[\dots]the art of enhancing the appearance and readability of a
document while exhibiting a minimum degree of visual obtrusion.
It is concerned with what happens between or at the margins of characters,
words or lines. Whereas the macro-typographical aspects of a document
(i.e., its layout) are clearly visible even to the untrained eye,
micro-typographical refinements should ideally not even be recognisable.
That is, you may think that a document looks beautiful, but you
might not be able to tell exactly why: good micro-typographic practice tries to
reduce all potential irritations that might disturb a reader.\punckern\endnote{%
R Schlicht,
\textit{The microtype package}
(v2.7a, January 14, 2018), 4}
\end{quote}

In \LaTeX{}, microtypography is controlled with the
\texttt{microtype} package.
Its use is automatic---for the vast majority of documents, you should add
\begin{leftfigure}
\begin{lstlisting}
\usepackage{microtype}
\end{lstlisting}
\end{leftfigure}
to your preamble and carry on---but let's take a brief look at what the package
does.

\section{Character protrusion}

By default, \LaTeX{} justifies lines between perfectly straight
left and right margins.
This is the obvious choice,
but falls victim to an annoying optical illusion:
lines ending in small glyphs---like periods, commas,
or hyphens---seem shorter than lines that
don't.\punckern\footnote{Many other optical illusions come up in typography.
For example, if a circle, a square, and a triangle
of equal heights are placed next to each other,
the circle and triangle look smaller than the square.
For this reason, round or pointed characters (like O and A) must
be made slightly taller than ``flat'' ones (such as H and T) for all
to appear the same height.\punckern\endnote{%
Jost Hochuli, \textit{Detail in typography}
(Éditions~\textsc{b}42, 2015),
18--19}}
\texttt{microtype} compensates by \introduce{protruding} these smaller glyphs
into the margins.

\section{Font expansion}

In order to to help \LaTeX's justification algorithm build paragraphs with
more even spacing and fewer hyphenated lines,
\texttt{microtype} can stretch characters horizontally.
You might think that distorting the type this way would be immediately
noticeable,
but you're reading a book that does so on every page!
This effect, called \introduce{font expansion},
is applied \emph{very} slightly---by default,
character widths are altered by no more than two percent.\punckern\footnote{%
Of course, you can use package options to change this limit,
or disable the feature entirely.}

This feature isn't currently available for \XeLaTeX{}.
You'll need to use \LuaLaTeX{} if you'd like to take advantage of it.

\section{What next?}

As always, see the package manual for ways to tweak these features.
\texttt{microtype} is capable of a few other tricks,
but several only work on older \LaTeX{} engines.\punckern\footnote{i.e., pdf\TeX}
Those we care about---such as letterspacing---can be handled with
\texttt{fontspec} or other packages.
